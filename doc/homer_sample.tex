%&program=xelatex
%&encoding=UTF-8 Unicode

\documentclass[12pt,letterpaper,twoside,openany,showidx]{book}
\usepackage[silent]{fontspec}
\usepackage{xltxtra}
\usepackage{polyglossia}
\setdefaultlanguage{greek}
\newICUfeature{Contextual}{on}{+calt}
\defaultfontfeatures{Mapping=tex-text,Script=Greek,Contextual=on}
\newcommand{\hlig}[1]{{\addfontfeature{Ligatures=Historic}{#1}}}
\newcommand{\salt}[1]{{\addfontfeature{Style=Alternate}{#1}}}
\setromanfont{Foulis Greek}
\begin{document}
\noindent\Large\addfontfeature{Ligatures=Historic} ΜΗ̃ΝΙΝ
ἄειδε, ΘΕᾺ, Πηληϊάδεω ἈΧΙΛΗ̃ΟΣ\\
Οὐλομένην, ἣ μυρί᾽ Ἀχαιοῖς ἄλγε᾽ ἔθηκε·\\
Πολλὰς δ᾽ ἰφθίμους ψυχὰς ἄϊδι προΐαψεν\\
Ἡρώων, αὐτοὺς δ᾽ ἑλώρια τεῦχε κύνεσσιν\\
Οἰωνοῖσί τε πᾶσι· Διὸς δ᾽ ἐτελείε\salt{τ}ο βουλή·\\
Ἐξ οὗ δὴ τὰ πρῶτα διαστήτην ἐ\salt{ρ}ίσαν\salt{τ}ε\\
Ἀτρεΐδης τε, ἄναξ ἀνδρῶν ϗ δῖος Ἀχιλλεύς.\\
    Τίς τ᾽ ἄρ σφωε \salt{θ}εῶν ἔριδι ξυνέηκε μάχεσθαι;\\
Λητοῦς καὶ Διὸς υἱός· ὃ γὰρ βασιλῆϊ χολωθεὶς\\
Νοῦσον ἀνὰ στρατὸν ὄρσε κακήν· ὀλέκον\salt{τ}ο δὲ λαοί·\\
Οὕνεκα τὸν Χρύσην ἠτίμησ᾽ ἀρητῆρα\\
Ἀτρεΐδης· ὃ γὰρ ἦλθε \salt{θ}οὰς ἐπὶ νῆας Ἀχαιῶν\\
Λυσόμενός τε \salt{θ}ύγα\salt{τρα φ}έρων τ᾽ ἀπερείσι᾽ ἄποινα,\\
Στέμμα\salt{τ᾽} ἔχων ἐν χερσὶν ἑκη\salt{β}όλου Ἀπόλλωνος,\\
Χρυσέῳ ἀνὰ σκήπ\salt{τ}ρῳ καὶ ἐλίσσε\salt{τ}ο πάν\salt{τ}ας Ἀχαιούς,\\
Ἀτρεΐδα δὲ μάλιστα δύω, κοσμήτορε λαῶν·\\
    Ἀτρεΐδαι τε, καὶ ἄλλοι ἐϋκνήμιδες Ἀχαιοί,\\
Ὑμῖν μὲν \salt{θ}εοὶ δοῖεν Ὀλύμπια δώμα\salt{τ᾽} ἔχοντες\\
Ἐκπέρσαι Πριάμοιο πόλιν, εὖ δ᾽ οἴκαδ᾽ ἱκέσθαι:\\
Παῖδα δέ μοὶ λύσαι\salt{τ}ε \salt{φ}ίλην, τὰ δ᾽ ἄποινα δέχεσθε,\\
Ἁζόμενοι Διὸς υἱὸν ἑκη\salt{β}όλον Ἀπόλλωνα.\\
    Ἔνθ᾽ ἄλλοι μὲν πάντες ἐπευφήμησαν Ἀχαιοὶ,\\
Αἰδεῖσθαί \salt{θ}᾽ ἱερῆα, ϗ ἀγλαὰ δέχθαι ἄποινα·\\
Ἀλλ᾽ οὐκ Ἀτρεΐδῃ Ἀγαμέμνονι ἥνδανε \salt{θ}υμῷ,\\
Ἀλλὰ κακῶς ἀφίει, κρα\salt{τ}ερὸν δ᾽ ἐπὶ μῦθον ἔτελλε·\\
    Μή σε, γέρον κοίλῃσιν ἐγὼ παρὰ νηυσὶ κιχείω\\
Ἢ νῦν δηθύνον\salt{τ᾽} ἢ ὕστερον αὖτις ἰόν\salt{τ}α,\\
Μή νύ τοι οὐ χραίσμῃ σκῆπ\salt{τ}ρον ϗ στέμμα \salt{θ}εοῖο.\\
Τὴν δ᾽ ἐγὼ οὐ λύσω, πρίν μιν ϗ γῆρας ἔπεισιν,\\
Ἡμετέρῳ ἐνὶ οἴκῳ ἐν Ἄργεϊ τηλόθι πάτρης\\
Ἱστὸν ἐποιχομένην, ϗ ἐμὸν λέχος ἀν\salt{τ}ιόωσαν·\\
Ἀλλ᾽ ἴθι μή μ᾽ ἐρέθιζε σαώτερος ὥς κε νέηαι.\\
    Ὣς ἔφα\salt{τ᾽}· ἔδδεισεν δ᾽ ὃ γέρων, ϗ ἐπείθε\salt{τ}ο μύθῳ·\\
Βῆ δ᾽ ἀκέων παρὰ \salt{θ}ῖνα πολυφλοίσ\salt{β}οιο \salt{θ}αλάσσης·\\
Πολλὰ δ᾽ ἔπει\salt{τ᾽} ἀπάνευθε κιὼν ἠρᾶθ᾽ ὃ γεραιὸς\\
Ἀπόλλωνι ἄνακ\salt{τ}ι, τὸν ἠΰκομος τέκε Λητώ·\\
    Κλῦθί μευ Ἀργυρότοξ᾽, ὃς Χρύσην ἀμφι\salt{βέβ}ηκας\\
Κίλλάν τε ζαθέην Τενέδοιό τε ἶφι ἀνάσσεις,\\
Σμινθεῦ εἴ πο\salt{τ}έ τοι χαρίεν\salt{τ᾽} ἐπὶ νηὸν ἔρεψα,\\
Ἢ εἰ δή πο\salt{τ}έ τοι κατὰ πίονα μηρί᾽ ἔκηα\\
Ταύρων ἠδ᾽ αἰγῶν, τὸ δέ μοι κρήηνον ἐέλδωρ·\\
Τίσειαν Δαναοὶ ἐμὰ δάκρυα σοῖσι βέλεσσιν.\\
    Ὣς ἔφα\salt{τ᾽} εὐχόμενος· τοῦ δ᾽ ἔκλυε Φο\salt{ῖβ}ος Ἀπόλλων,\\
Βῆ δὲ κα\salt{τ᾽} oὐλύμποιο καρήνων χωόμενος κῆρ,\\
Τόξ᾽ ὤμοισιν ἔχων ἀμφηρε\salt{φ}έα τε φαρέτρην·\\
Ἔκλαγξαν δ᾽ ἄρ᾽ ὀϊστοὶ ἐπ᾽ ὤμων χωομένοιο,\\
Αὐτοῦ κινηθέν\salt{τ}ος· ὃ δ᾽ ἤϊε νυκ\salt{τ}ὶ ἐοικώς.\\
Ἕζε\salt{τ᾽} ἔπει\salt{τ᾽} ἀπάνευθε νεῶν, με\salt{τὰ} δ᾽ ἰὸν ἕηκε·\\
Δεινὴ δὲ κλαγγὴ γένε\salt{τ᾽} ἀργυρέοιο βιοῖο.\\
Οὐρῆας μὲν πρῶτον ἐπῴχε\salt{τ}ο ϗ κύνας ἀργούς,\\
Αὐτὰρ ἔπει\salt{τ᾽} αὐτοῖσι βέλος ἐχεπευκὲς ἐφιεὶς\\
Βάλλ᾽· αἰεὶ δὲ πυραὶ νεκύων καίον\salt{τ}ο \salt{θ}αμειαί.\\
Ἐννῆμαρ μὲν ἀνὰ στρατὸν ᾤχετο κῆλα \salt{θ}εοῖο,\\
Τῇ δεκάτῃ δ᾽ ἀγορὴν δὲ καλέσσα\salt{τ}ο λαὸν Ἀχιλλεύς·\\
Τῷ γὰρ ἐπὶ φρεσὶ \salt{θ}ῆκε \salt{θ}εὰ λευκώλενος Ἥρη·\\
Κήδε\salt{τ}ο γὰρ Δαναῶν, ὅτι ῥα \salt{θ}νήσκον\salt{τ}ας ὁρᾶτο.\\
Οἳ δ᾽ ἐπεὶ οὖν ἤγερθεν, ὁμηγερέες τ᾽ ἐγένον\salt{τ}ο,\\
Τοῖσι δ᾽ ἀνιστάμενος μετέφη πόδας ὠκὺς Ἀχιλλεύς·\\
    Ἀτρεΐδη νῦν ἄμμε παλιμπλα\salt{γ}χθέντας ὀΐω\\
Ἂψ ἀπονοστήσειν, εἴ κεν \salt{θ}άνα\salt{τ}όν γε φύγοιμεν·\\
Εἰ δὴ ὁμοῦ πόλεμός τε δαμᾷ ϗ λοιμὸς Ἀχαιούς·\\
Ἀλλ᾽ ἄγε δή τινα μάντιν ἐρείομεν, ἢ ἱερῆα,\\
Ἢ καὶ ὀνειροπόλον, καὶ γάρ τ᾽ ὄναρ ἐκ Διός ἐστιν,\\
Ὅς κ᾽ εἴποι ὅ τι τόσσον ἐχώσα\salt{τ}ο Φο\salt{ῖβ}ος Ἀπόλλων·\\
Εἴτ᾽ ἄρ᾽ ὅ γ᾽ εὐχωλῆς ἐπιμέμφεται ἠδ᾽ ἑκατόμ\salt{β}ης·\\
Αἴ κέν πως ἀρνῶν κνίσσης αἰγῶν τε τελείων\\
Βούλε\salt{τ}αι ἀντιάσας ἡμῖν ἀπὸ λοιγὸν ἀμῦναι.
\end{document}
